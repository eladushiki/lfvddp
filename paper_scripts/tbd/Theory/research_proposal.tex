\documentclass{article}

\usepackage[utf8]{inputenc}
\usepackage{amsmath}
\usepackage{amssymb}
\usepackage{amsfonts}
\usepackage{hyperref}
\usepackage{graphicx}
\usepackage[english]{babel} % enables hyphenation
\usepackage{microtype} % improves spacing overall


\title{Research Proposal}
\author{Elad Kliger}
\date{\today}

\begin{document}

\maketitle

\section{Scientific Background}

The Standard Model (SM) of particle physics is a well established theory that describes the very basic building blocks of matter and their interactions. It account for all observed matter and antimatter, and all known forces except for gravity. The theory has been thoroughly tested in many experiments.
This being said, it is clear that the Standard Model is not a complete theory of nature, since there are certain phenomena it does not account for. Neutrino masses and dark matter are two of them. This motivates the development of Beyond Standard Model (BSM) theories and triggers searches for signatures originating from such models.

Most theoretically motivated extensions of the SM were or are being searched for, with majority of them already excluded. Thus, at current times, there are no strong theoretical motivation to prefer the search for evidence for one specific BSM theory over another.

Despite this huge effort, the majority of the data produced by the Large Hadron Collider (LHC) have not been thoroughly investigated. Using current search techniques, which examine a specific theoretically possible extension of the model at a time, covering all possible scenarios is impossible. For it was not yet proven otherwise, BSM physics could be hidden within reach.

Recent works suggest a new method, the Data Directed Paradigm (DDP)\cite{LastDDPpaper}, to invest in detecting various unexplained deviations from the SM prediction. To examine the vast amounts of data, a machine is used, before investing expensive physicist-years in building a model that explains them.

Lepton Flavour Universality (LFU) is an an accidental symmetry of the SM, broken only by the Yukawa interactions. These break the universal symmetry to a $U(1)_{e} \times U(1)_{\mu} \times U(1)_{\tau}$ that is lepton flavor conserving. Consequently, any two datasets that differ only by the flavor of lepton (neglecting phase space and detector effects, along with Yukawa interactions), should have the same statistics\cite{LeptonFlavorViolationHiggsDecays}. We will examine pairs of datasets that differ by replacing electrons that appear in one's cuts with a muon. Thus, any observed asymmetry in them is an indication for BSM physics. 

We suggest utilizing this symmetry of nature to scan through all available data, looking for deviations.

\section{Goal}
Our goal is to discover BSM physics, if exists, that violate LFU.

We'll do so by building a tool to discover said violations in a systematic way, estimate its sensitivity using simulated datasets before applying it to real experimental data.

\section{Existing Tools}
Previous research suggests tools and methods for performing such analysis. Previous studies\cite{Italian no 1 - Learning New Physics from a Machine}\cite{Italian no 2 - Learning New Physics from an Imperfect Machine} lay the foundations of phrasing the problem of determining whether two data sets stem from the same underlying statistics, as well as using a neural network (NN) for the process.

Further studies\cite{LastDDPpaper} suggest applying the mechanism to look for broken LU in pairs of datasets.

%We would like to improve said tools to take into account more real world phenomena and estimate their predictive power.

\section{Proposal}
Our study would take existing computational tools and improve them in the following ways:

\begin{enumerate}
    \item Setting the machinery to handle datasets with multiple physical observables that describe each event.
    \item Applying simulated detector effects that may create asymmetry, and cope for them.
\end{enumerate}

The performance of our prediction would be compared to an "ideal" profile likelihood test\cite{Cowan and Eilam PL formulation}, having known the exact number of signal and background events in each simulation.

\textcolor{red}{Will continue from here}

\subsection{Preliminary Studies}
We were able to reproduce the results of the previous study in ref \cite{LastDDPpaper} and train a NN with comparative results.

Our first new results include modeling a detector effect, which affects two datasets that are otherwise summetric. We estimate the effect with some induced errors, correct for it and create an approximation for the true datasets.

\begin{figure}[h]
    \begin{minipage}{0.48\textwidth}
        \centering
        \includegraphics[width=\textwidth,trim=0 18pt 0 0,clip]
        {../research_proposal/run_at_20250620_231054_of_single_train.py_on_commit_ba286_pid_2578962/A_data_process_plot_-6195891144235862255.png}
    \end{minipage}\hfill
    \begin{minipage}{0.48\textwidth}
        \centering
        \includegraphics[width=\textwidth,trim=0 18pt 0 0,clip]
        {../research_proposal/run_at_20250620_231054_of_single_train.py_on_commit_ba286_pid_2578962/B_data_process_plot_-6195891144235862255.png}
    \end{minipage}
    \caption{The process of generating a dataset, distributed by some arbitrary physical observable of the events. Left: 10000 exponentially decaying probability background events with 50 signal events distributed around $param\_0=4$. Right: no signal events. In each graph three steps are shown: 1. The true dataset in blue. 2. The observed dataset with missing events due to a non perfect detector efficiency. 3. The approximation for the original dataset made from the observed dataset and non-perfect knowledge of the detector efficiency.}
    \label{fig:data_process_plot}
\end{figure}

\begin{thebibliography}{99}
    \bibitem{LastDDPpaper}
    Bressler, Shikma, Inbar Savoray, and Yuval Zurgil.
    \textit{‘Learning New Physics from Data -- a Symmetrized Approach’}. arXiv, 2024. \url{https://doi.org/10.48550/ARXIV.2401.09530}.

    \bibitem{LeptonFlavorViolationHiggsDecays}
    Bressler, Shikma, Avital Dery, and Aielet Efrati. ‘Asymmetric Lepton-Flavor Violating Higgs Boson Decays’. Physical Review D 90, no. 1 (22 July 2014): 015025. https://doi.org/10.1103/PhysRevD.90.015025.

    \bibitem{Italian no 1 - Learning New Physics from a Machine}
    D’Agnolo, Raffaele Tito, and Andrea Wulzer. ‘Learning New Physics from a Machine’. Physical Review D 99, no. 1 (8 January 2019): 015014. https://doi.org/10.1103/PhysRevD.99.015014.

    \bibitem{Italian no 2 - Learning New Physics from an Imperfect Machine}
    Agnolo, Raffaele Tito d’, Gaia Grosso, Maurizio Pierini, Andrea Wulzer, and Marco Zanetti. ‘Learning New Physics from an Imperfect Machine’. arXiv, 26 November 2021. https://doi.org/10.48550/arXiv.2111.13633.

    \bibitem{Cowan and Eilam PL formulation}
    Cowan, Glen, Kyle Cranmer, Eilam Gross, and Ofer Vitells. ‘Asymptotic Formulae for Likelihood-Based Tests of New Physics’. The European Physical Journal C 71, no. 2 (February 2011): 1554. https://doi.org/10.1140/epjc/s10052-011-1554-0.
\end{thebibliography}

\end{document}

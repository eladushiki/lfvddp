\documentclass[../thesis.tex]{subfiles}

\makeatletter
\@ifundefined{preamble@loaded}{
    \def\preamble@loaded{}

\usepackage[utf8]{inputenc}
\usepackage{amsmath}
\usepackage{amssymb}
\usepackage{bm}
\usepackage{amsfonts}
\usepackage{hyperref}
\usepackage{graphicx}
\usepackage{geometry}
\usepackage[english]{babel} % enables hyphenation
\usepackage{microtype} % improves spacing overall
\usepackage{physics}
\usepackage{xcolor}

\usepackage{subfiles} % Best loaded last
% =============================================================================
% COMMON DEFINITIONS AND SHORTCUTS
% =============================================================================
% This file contains LaTeX function definitions and shortcuts to be included
% in other tex files. Include with: % =============================================================================
% COMMON DEFINITIONS AND SHORTCUTS
% =============================================================================
% This file contains LaTeX function definitions and shortcuts to be included
% in other tex files. Include with: \input{../common/definitions.tex}

% =============================================================================
% NOTATIONS
% =============================================================================

%% Data sets
\newcommand{\dataset}[1]{{\mathbf{#1}}}         % A dataset
\newcommand{\true}[1]{{\dataset{#1}}_{\text{true}}}  % A true dataset
\newcommand{\meas}[1]{{\dataset{\tilde{#1}}}}   % A measured dataset
\newcommand{\reco}[1]{{\dataset{\dot{#1}}}}     % A reconstructed dataset

%% Detecting events
\newcommand{\detector}{\mathrm{Det}}            % Detector function

%% Hypotheses
%%% Distributions
\newcommand{\dist}[1]{{n_{\dataset{#1}}(x)}}    % Any distribution of x
\newcommand{\refdist}[1][x]{n_{\mathcal{R}}({#1})}  % Reference distribution
%% Likelihood
\newcommand{\likelihood}{\mathcal{L}}           % Likelihood function
\newcommand{\llikelihood}{\log\mathcal{L}}      % Log-likelihood function

% =============================================================================
% CUSTOM COLORS (if needed)
% =============================================================================

% Define custom colors for highlighting, etc.
% \definecolor{...}{RGB}{...}

% =============================================================================
% END OF DEFINITIONS
% =============================================================================


% =============================================================================
% NOTATIONS
% =============================================================================

%% Data sets
\newcommand{\dataset}[1]{{\mathbf{#1}}}         % A dataset
\newcommand{\true}[1]{{\dataset{#1}}_{\text{true}}}  % A true dataset
\newcommand{\meas}[1]{{\dataset{\tilde{#1}}}}   % A measured dataset
\newcommand{\reco}[1]{{\dataset{\dot{#1}}}}     % A reconstructed dataset

%% Detecting events
\newcommand{\detector}{\mathrm{Det}}            % Detector function

%% Hypotheses
%%% Distributions
\newcommand{\dist}[1]{{n_{\dataset{#1}}(x)}}    % Any distribution of x
\newcommand{\refdist}[1][x]{n_{\mathcal{R}}({#1})}  % Reference distribution
%% Likelihood
\newcommand{\likelihood}{\mathcal{L}}           % Likelihood function
\newcommand{\llikelihood}{\log\mathcal{L}}      % Log-likelihood function

% =============================================================================
% CUSTOM COLORS (if needed)
% =============================================================================

% Define custom colors for highlighting, etc.
% \definecolor{...}{RGB}{...}

% =============================================================================
% END OF DEFINITIONS
% =============================================================================


}{}
\makeatother


\begin{document}

\section{Event Processing}
\subsection{Building an Observed Dataset}

During a HEP experiment, large amount of collision events occur inside the detector. The original number of such events in a dataset is denoted by $\true{N}$.

When detected, some $j$ physical attributes of each event are measured, e.g., energy, momentum components and angles. Let each event \( \bar{x_i} \in \mathbb{R}^k \equiv \{\phi^j_i\}\) be a point in \( k \)-dimensional feature space. Each dataset \(\mathbf{A} \in \mathbb{R}^{\true{N} \times k}\) consists of such events.

Due to limited detector efficiency and selection effects, not all events that occur inside the detector are observed. Let $\eta(\bar{x})$ be the probability with which an event is detected, and hence appears in the observed dataset.

We define the filtering process \(\detector(\eta, \{\bar{x_i}\})\) that stochastically removes events:
\[
\detector(\eta, \{\bar{x_i}\}) \equiv \{ \bar{x_i} \mid \text{each } \bar{x_i} \text{ is retained with probability } \eta(\bar{x_i}) \}.
\]

The observed (filtered) dataset is then:
\[
\meas{A} = \detector(\eta, \dataset{A})
\]

with $|\meas{A}| = N_\meas{A}$, the observed number of events.

Our first goal is to estimate \( \true{N} \) from the observed \( N_\meas{A} \), correcting for detector efficiency effects and propagating uncertainties associated with the efficiency estimation.

\subsection{Modelling our Knowledge of the Detector Efficiency}

Now expressing the (true) efficiency functions as:

\begin{align*}
    \eta(\bar{x_i}) &: \mathbb{R}^k \to [0,1] \equiv \prod_{j=1}^k \eta^j(\phi_i^j)
\end{align*}

with

\begin{align} \label{eq:efficiency_j_explicit_form}
    \theta^j(\phi^j) &: \mathbb{R} \to [0,1] \equiv e^{\nu^j_m}\eta^j(\phi^j)
\end{align}

with $\theta^i$ representing our knowledge of the efficiency of each bin. This means, we model for an efficiency function that varies from our understanding of it through usage of nuisance parameters $\nu^j_m$. While considered small for our formulation, for the time being - are unconstrained.

Datasets statistics are being binned before they can be analyzed. Given that the detectable range is divided into $m^j$ bins along each dimension, we consider the efficiency of the detector to be a multiplicative constant in each. We further claim and model $\eta$ and $\theta$ to be evaluated at a per-bin basis. This gives taht we need exactly $j \times m$ nuisance parameters, $\nu^j_m$, to fully capture the uncertainty effect.

\subsection{Efficiency Compensation via Event Weights}

To reconstruct the statistics of the original datasets, we assign weights:

\begin{equation}
\label{eq:def_weights}
\text{for } x_i \in \tilde{\mathbf{A}}, \quad w_i := \frac{1}{\theta(x_i)}
\end{equation}

This gives rise to the efficiency-compensated reconstructed datasets:

\[
\dot{\mathbf{A}} := \{ (x_i, w_i) \mid x_i \in \tilde{\mathbf{A}} \}
\]

The estimated total event counts become:
\[
N_\reco{A} = \sum_{x_i \in \meas{A}} w_i \approx \true{N}
\]

Note that this expression includes our knowledge of the efficiency and our error, throught the $\eta$'s in the expression.

\subsection{Proposition (Unbiasedness of the weighted estimator).} \label{prop:unbiasedness_of_weights}
Let \(\{x_i\}_{i=1}^{\true{N}}\) be the set of true collected events, each detected independently with probability \(\eta(x_i)\). Then:
\[
\mathbb{E}\!\left[ N_\reco{A} \right] = \true{N}
\]
\emph{Proof.}
For a single event \(x_i\), the contribution to the sum is \(\frac{1}{\theta(x_i)}\) if detected and \(0\) otherwise. Its expectation is:
\begin{align*}
\eta(x_i) \cdot \frac{1}{\theta(x_i)} + (1-\eta(x_i))\cdot 0 &= \\
= e^{\nu(x_i)}\theta(x_i) \cdot \frac{1}{\theta(x_i)} &= e^{\nu(x_i)}
\end{align*}

% If we leave this formulation take the proof from https://statproofbook.github.io/P/norm-mgf.html
% It says that E(x) = e^(\mu x + 1/2 sigma^2 x^2)

Given that the expectation value of the exponentiated nuisances is 0, summing over all \(\true{N}\) events yields the result. \(\square\)

\end{document}